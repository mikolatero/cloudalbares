\documentclass[serif,9pt]{beamer}
\setbeamertemplate{navigation symbols}{}


\usetheme{Warsaw}

\usepackage[latin1]{inputenc}
\usepackage[spanish]{babel}

\beamersetuncovermixins{\opaqueness<1>{25}}{\opaqueness<2->{15}}

\begin{document}

\title[Virtualizaci�n en la educacion \ldots ]{Plataformas de virtualizacion para equipos en centros educativos}  
\author{Francisco Jos� Marin Cano \\ Jos� Maria Alcaraz Marin}
\institute[D.S.Y.C]{%
  Departamento T�cnico de Sistemas y Comunicaciones\\
  Cieza (Murcia)}
\date{Copyleft \copyright{} 2013. Reproducci�n permitida bajo los \\
      t�rminos de la licencia de documentaci�n libre GNU.}

\section{Inicio}
\begin{frame}
\titlepage
\begin{center}
\includegraphics[scale=0.3]{albares1.jpg}
\end{center} 
\end{frame}

\section{Indice}
\begin{frame}\frametitle{Contenido}\tableofcontents
\end{frame} 


\section{Implantacion en IES Los Albares} 

\subsection{Temporalizacion}

\begin{frame}\frametitle{Linea de Tiempo} 
\begin{itemize}

\item 
\pause \bigskip

\item 
\pause \bigskip

\item 

\end{itemize}

\end{frame}


\subsection{Sistemas utilizados}

\begin{frame}\frametitle{OpenNebula} 



\end{frame}
\begin{frame}\frametitle{OpenStack}

\end{frame}
\begin{frame}\frametitle{StackOps}

\end{frame}
\begin{frame}\frametitle{oVirt}

\end{frame}
\begin{frame}\frametitle{DevStack}

\end{frame}

\subsection{Estado Actual}
\begin{frame}\frametitle{�Como se encuentra actualmente el proyecto?}

\end{frame}



\section{Infraestructura necesaria para Virtualizaci�n}
\subsection{?}
\begin{frame}\frametitle{texto}

\end{frame}
\subsection{?}
\begin{frame}\frametitle{texto}

\end{frame}
\subsection{?}
\begin{frame}\frametitle{texto}

\end{frame}

\end{document}

